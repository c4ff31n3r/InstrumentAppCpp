\documentclass[14pt]{extarticle}
\usepackage[a4paper, left=30mm, top=20mm, bottom=20mm, right=15mm]{geometry} 
\usepackage[T1, T2A]{fontenc}
\usepackage[utf8]{inputenc}
\usepackage[russian]{babel}
\usepackage{cmap}
\usepackage{textcomp}
\usepackage{indentfirst}
\usepackage{amssymb}
\usepackage{amsmath}
\usepackage{graphicx}
\usepackage{minted}
\usepackage{fvextra}
\usepackage{titlesec}
\usepackage{longtable}
\usepackage[labelsep=endash]{caption}
\usepackage{tocloft}
\usepackage{float}
\usepackage{newunicodechar}
\usepackage{multirow}
\usepackage{pgf-umlcd}

\sloppy
\clubpenalty=10000
\widowpenalty=10000

\titleformat{\section}
    {\normalfont\fontsize{16}{18}\bfseries}
    {\thesection}
    {0.25em}
    {}
    
\titleformat{\subsection}
    {\normalfont\fontsize{14}{16}\bfseries}
    {\hspace{1em}\thesubsection}
    {0.25em}
    {}

\begin{document}
\newpage\thispagestyle{empty}
\renewcommand{\figurename}{Рисунок}
\begin{center}
МИНИСТЕРСТВО НАУКИ И ВЫСШЕГО ОБРАЗОВАНИЯ \\
РОССИЙСКОЙ ФЕДЕРАЦИИ \\[2ex]
ФЕДЕРАЛЬНОЕ ГОСУДАРСТВЕННОЕ БЮДЖЕТНОЕ ОБРАЗОВАТЕЛЬНОЕ
УЧРЕЖДЕНИЕ ВЫСШЕГО ОБРАЗОВАНИЯ \\
«ВЯТСКИЙ ГОСУДАРСТВЕННЫЙ УНИВЕРСИТЕТ» \\[2ex]
Факультет автоматики и вычислительной техники \\[2ex]
Кафедра электронных вычислительных машин \\
\vfill
\mbox{}\hfill
\begin{tabular}{lp{5em}r}
Дата сдачи на проверку: \\
<<\rule[-5pt]{25pt}{0.5pt}>> \rule[-5pt]{70pt}{0.5pt} 2025 г. \\[2ex]
Проверено: \\
<<\rule[-5pt]{25pt}{0.5pt}>> \rule[-5pt]{70pt}{0.5pt} 2025 г.\\
\end{tabular}

\vfill
\textbf{ООП в C++} \\
Отчет по лабораторной работе № 1 \\
по дисциплине \\
<<Объектно-ориентированное программирование>> \\

\vfill
Выполнил студент гр. ИВТб-2303-05-00 \hfill \rule[-5pt]{0.22\linewidth}{1pt} /Крупица Р.А./ \\
\hfil {\footnotesize (Подпись)}\\[2ex]    
Руководитель \hfill \rule[-5pt]{0.22\linewidth}{1pt} /Шмакова Н.А./ \\
\hfil {\footnotesize (Подпись)}\\[2ex]

Работа защищена \hfill <<\rule[-5pt]{25pt}{0.5pt}>> \rule[-5pt]{70pt}{0.5pt} 2025 г.
\vfill
Киров 2025
\end{center}
\newpage

\section{Цель работы}
Закрепить навыки объектно-ориентированного программирования, научиться использовать наследование, полиморфизм и виртуальные функции на примере иерархии классов, описывающих музыкальные инструменты.

\section{Задание}
Разработать и реализовать на языке C++ консольное приложение, моделирующее взаимодействие с различными музыкальными инструментами. Предусмотреть базовый класс и производные классы (гитара, пианино), каждый из которых реализует специфичное поведение.

\section{Описание реализации}
В рамках лабораторной работы был создан базовый класс \texttt{MusicalInstrument}, включающий основные характеристики музыкальных инструментов — имя и количество струн/клавиш. В классе реализованы виртуальные методы для описания, настройки, игры и уникальной особенности инструмента.

От базового класса унаследованы:
\begin{itemize}
    \item \textbf{Guitar} — содержит дополнительную информацию о типе гитары (акустическая или электрическая) и наличие звукоснимателя;
    \item \textbf{Piano} — реализует свойства пианино или рояля, а также количество педалей;
\end{itemize}

В главной функции реализовано меню взаимодействия: добавление, удаление и выбор инструмента с дальнейшим вызовом методов.

\newpage
\section{Описание свойств и методов классов}

\subsection*{Базовый класс \texttt{MusicalInstrument}}

\textbf{Свойства:}

\begin{itemize}
\item \texttt{name}~--- имя инструмента;
\item \texttt{stringsCount}~--- количество струн инструмента.
\end{itemize}

\textbf{Методы:}

\begin{itemize}
    \item \texttt{play()}~--- выводит сообщение о том, что инструмент играет.
    \item \texttt{tune()}~--- имитирует настройку инструмента.
    \item \texttt{describe()}~--- выводит общее описание инструмента (имя и количество струн/клавиш).
    \item \texttt{specialFeature()}~--- демонстрирует уникальную особенность инструмента.
    \item \texttt{setName(name)}~--- устанавливает новое имя инструмента.
    \item \texttt{getName()}~--- возвращает текущее имя инструмента.
    \item \texttt{setStringsCount(count)}~--- задаёт количество струн/клавиш.
    \item \texttt{getStringsCount()}~--- возвращает текущее количество струн/клавиш.
\end{itemize}

\subsection*{Производный класс \texttt{Guitar}}

\textbf{Свойства:}

\begin{itemize}
\item \texttt{guitarType}~--- тип гитары (акустическая/электрическая);
\item \texttt{hasPickup}~--- наличие звукоснимателя.
\end{itemize}

\textbf{Методы:}

\begin{itemize}
    \item \texttt{strum()}~--- имитирует перебор струн гитары.
    \item \texttt{changeStrings()}~--- показывает процесс замены струн на гитаре.
    \item \texttt{describe()}~--- переопределяет базовый метод, добавляя информацию о типе гитары и наличии звукоснимателя.
    \item \texttt{specialFeature()}~--- переопределяет базовый метод, уточняя, что гитара способна играть как аккорды, так и мелодии.
\end{itemize}

\subsection*{Производный класс \texttt{Piano}}

\textbf{Свойства:}

\begin{itemize}
\item \texttt{pedalCount}~--- количество педалей;
\item \texttt{isGrand}~--- является ли пианино роялем (у рояля струны и механика расположены горизонтально. Молоточки возвращаются под действием силы тяжести, что делает игру более чувствительной и быстрой). 
\end{itemize}

\textbf{Методы:}

\begin{itemize}
    \item \texttt{pressPedal()}~--- имитирует нажатие педали.
    \item \texttt{cleanKeys()}~--- отображает процесс очистки клавиш.
    \item \texttt{describe()}~--- переопределяет базовый метод, выводит тип (рояль или пианино), количество клавиш и педалей.
    \item \texttt{specialFeature()}~--- переопределяет базовый метод, поясняет возможность одновременного исполнения мелодии и аккомпанемента.
\end{itemize}

\section{Диаграмма классов}

\begin{figure}[H]
\centering
\begin{tikzpicture}
    % Базовый класс
    \begin{class}[text width=6cm]{MusicalInstrument}{0,0}
        \attribute{name: string}
        \attribute{stringsCount: int}
        \operation{play(): void}
        \operation{tune(): void}
        \operation{describe(): void}
        \operation{specialFeature(): void}
        \operation{setName(name: string): void}
        \operation{getName(): string}
        \operation{setStringsCount(count: int): void}
        \operation{getStringsCount(): int}
    \end{class}
    
    % Класс Guitar
    \begin{class}[text width=6cm]{Guitar}{8,3}
        \inherit{MusicalInstrument}
        \attribute{guitarType: string}
        \attribute{hasPickup: bool}
        \operation{strum(): void}
        \operation{changeStrings(): void}
        \operation{describe(): void}
        \operation{specialFeature(): void}
    \end{class}
    
    % Класс Piano
    \begin{class}[text width=6cm]{Piano}{8,-3}
        \inherit{MusicalInstrument}
        \attribute{pedalCount: int}
        \attribute{isGrand: bool}
        \operation{pressPedal(): void}
        \operation{cleanKeys(): void}
        \operation{describe(): void}
        \operation{specialFeature(): void}
    \end{class}
\end{tikzpicture}
\caption{Диаграмма классов}
\end{figure}

\section{Вывод}
В результате выполнения лабораторной работы были достигнуты следующие результаты:
\begin{itemize}
    \item разработана иерархия классов с использованием принципов ООП;
    \item освоены техники наследования, переопределения методов и полиморфизма;
    \item реализована консольная система управления коллекцией музыкальных инструментов;
    \item создана UML-диаграмма, отражающая архитектуру программы.
\end{itemize}

Полученные знания и навыки будут полезны при дальнейшем изучении системного проектирования и архитектуры программного обеспечения.

\end{document}
